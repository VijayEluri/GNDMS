
\section {Configuration of \ldots STUFF}
Several objects in GNDMS can/must be configured to work properly. This
configuration can mostly happen during run-time throw a provided
Monitor interface.

Using this interface so called \textit{setup actions} can be invoked.
A setup action is usually dedicated to a class, can either create
instances of this class, which represent an extension to the GNDMS
system or alter so already existing instances. 

For some classes this actions are actually split in two actions a
create action which sets up a new instance and stores it in the
underlying database\footnote{The ability to store the instance in the
database must be provided by the class, and doesn't come out of the
box} and a setup action which configures the instance.

Both actions are highly depending on the class they configure, and
must be changed in cases some configurable parameter of a class
changes. This actions can be equipped with an on-line help, which can
be used by the uses to learn what are the action is capable of.
Programmatically this information is added by annotation the action
class itself and its instance variables.

A more generic configuration mechanism, which is depending on the
action framework, is introduced in the next section.

\subsection{The Configlet Framework}

A configlet provides a place where the settings of configurable
objects can be stored. Furthermore a configlet can able to update is
associated object when a change in the configuration occurs.
Apart from the data a configlet has a name which is used to identify
it within the application.

The configuration data contained in a configlet is stored in the
underlying database and in this way kept persistent when the
application is restarted.

There are several kinds of configlets in GNDMS: some which run in
separate threads and refresh their configuration data periodically,
other which are dedicated to concrete classes and are customized to
fit their special needs and other called \glqq Default-Configlets\grqq
which offer a generic map to store an \textit{option $\rightarrow$
value association}.

As mentioned above configlets can be associated to a class or a
singleton instance of a class. If the configlet is dedicated to a
class then it's usually passed to the places where objects of the
class are created this can happen using delegation or inject. Another
way is the \textit{ConfigletProvider}, which contains references to
all configlets existing in the application. This references can be
queried using the referenced configlets name. So whenever an instance
needs access to a configlet it can be supplied with the
\textit{configlet provider} ant uses it to acquire the desired
configlet. The GNDMS-System instance ensures that all of this
configlets are up-to-date and takes care for persisting them.

By default the configlet itself doesn't offer any functionality to
alter the configuration and require a source which provides the
configuration data, that is where setup-actions come into play. The
setup actions are intended to be called at runtime via the
GNDMS-Monitor.  They are supplied with the instance or class they
should setup, here a configlet derived class, and the configuration in
text form. The configuration is parsed so that their configlet can
understand it, than it's handed over to the configlet.

In order to utilize the configlet framework for new classes several
steps have to be performed. At first one must decide which configlet
functionality is required by the new class. If the configlet should
read data from a fluctuating source and update itself, than the
\texttt{RunnalbeConfiglet} it the way to go. On the other hand if a
static configuration for initial setup and rare reconfiguration is
required then the basic \texttt{DefaultConfiglet} should be the first
choice else if the configlet should to something exotic it would
be a good advice to started at the configlet interface and implement
it from scratch.

In the next section we describe the implementation of a custom
configlet which subclasses the \texttt{DefaultConfiglet}. The class we
want to configure has two parameters, both are of type integer.

When the configlet is instantiated for the first time its initialized
its \texttt{init} method is called, for the case that configuration
should be changed later on, an\texttt{update} method is
provided.\footnote{This applies for every configlet.}

Starting at the \texttt{DefaultConfiglet} we need to override this to
methods and to so that the changes passed to the class we want to
configure.
